\documentclass{article}
\usepackage[utf8]{inputenc}
\usepackage[spanish]{babel}
\usepackage{listings}
\usepackage{graphicx}
\graphicspath{ {images/} }
\usepackage{cite}

\begin{document}

\begin{titlepage}
    \begin{center}
        \vspace*{1cm}
            
        \Huge
        \textbf{Desafío de descripción}
            
        \vspace{0.5cm}
        \LARGE
        Llevar un objeto de una posición a otra
            
        \vspace{1.5cm}
            
        \textbf{Ricardo Echeverri Cano}
            
        \vfill
            
        \vspace{0.8cm}
            
        \Large
        Despartamento de Ingeniería Electrónica y Telecomunicaciones\\
        Universidad de Antioquia\\
        Medellín\\
        Marzo de 2021
            
    \end{center}
\end{titlepage}

\tableofcontents
\newpage
\section{Introducción}\label{intro}
Explicaremos el proceso detallado que se debe de realizar para mover dos objetos de una posición A a una posición B.

\section{Objetos a utilizar} \label{objetos}
\begin{itemize}
    \item Dos tarjetas con dimensiones igualas.
    \item Hoja de papel en perfecto estado y con una marca (*) en una de sus caras.
\end{itemize}
Este ejercicio se va a realizar en una superficie horizontal plana (mesa, etc.).

\section{Pasos a Seguir} \label{pasos}
\begin{itemize}
    \item Posición Inicial: Las tarjetas se encuentran sobre la superficie plana, una al lado de la otra, y sobre ellas se encuentra la hoja de papel con la cara marcada apuntando en dirección contraria a las tarjetas.
    \begin{enumerate}
        \item Procedemos a levantar la hoja de papel y descargarla sobre la superficie plana a un lado de la tarjeta que se encuentra al lado derecho, sin cambiar la dirección de la marca de la hoja de papel.
        \item Cogemos ambas tarjetas con una sola mano (su mano dominante).
        \item Sin descargarlas en una superficie las juntamos como si fueran una sola tarjeta de manera vertical.
        \item Luego apoyamos el lado inferior de las tarjetas juntas sobre la hoja de papel con la cara marcada, sin soltar las tarjetas de la mano.
        \item Con los dedos pulgar y medio agarramos los lados de las tarjetas en una posición media y el dedo índice lo apoyamos en el lado superior de las tarjetas. Todo esto se hará sin soltar las tarjetas.
        \item Luego los dedos pulgar y medio van a tratar de coger solo la tarjeta del lado de la mano dominante y la intentara separar de la otra tarjeta un poco, mientras que el dedo índice no permite que la parte superior de las tarjetas se separen.
    \end{enumerate}
    \item Posición Intermedia: En estos momentos se va a observar como las tarjetas van a formar una especie de triángulo isósceles.
    \begin{enumerate}
        \item Al tener los lados inferiores de las tarjetas separadas una de la otra y los lados superiores juntos, buscara una posición en la cual puedan sostenerse por sí solas sin necesidad de tenerlas con la mano. Puede alejar o acercar las tarjetas como se describió en el punto número (6).
        \item Cuando considere que las tarjetas están totalmente equilibradas la una de la otra procede a soltarlas.
    \end{enumerate}
    \item Posición Final: Ambas tarjetas con sus lados superiores juntos y sus lados inferiores separado formando un triángulo isósceles siendo la base la hoja de papel.
\end{itemize}
\end{document}
